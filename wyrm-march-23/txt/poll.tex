We at the ICSF committee were curious to learn how our society members and alumni would deal with The Apocalypse, so we conducted a devious and morally questionable social experiment in the form of……a survey!! This survey contained a number of questions to collect valuable data on what the state of the society and its members would be, should an Apocalyptic scenario arrive, as well as finding out some general opinions on The Apocalypse. Whilst it is statistically invalid to represent all members and alumni of ICSF with the 16 survey responders, we are going to do so anyway. 

Just over two thirds of ICSF survey responders believed they would not survive The Apocalypse, with an optimistic 31.3\% hoping to live to fight (or more likely binge the latest Marvel series) another day. Hopefully, we will be able to dig a chair, secretary and treasurer out of these survivors, allowing the society to continue to run. 

We asked responders to judge which style of apocalypse they believed would be the “coolest”, and the overall winner (with 31.3\% of the vote), was an Eldritch Apocalypse – the most typical example of this would be Lovecraft’s works, but a more modern take could be Gravity Falls’ “Weirdmageddon”. Unfortunately, no one thought this was actually likely to happen. The runner up, with 18.8\% of the votes, was a robot apocalypse, such as those portrayed in Terminator or the Matrix. In this case, ICSF would continue to run under the name “Extermination Society”, as Miranda, Chance and Norfolk (the 3 Daleks currently residing in the library), would replace the existing committee. 

Next, we asked whether ICSSCF survey responders had a zombie apocalypse plan in place, and whether they had any form of reserve apocalypse supplies or bunker. Whilst 68.8\% of responders had a zombie apocalypse plan, 56.3\% admitted that it was very bad. 18.8\% of responders selected the option “no of course not you are very weird”, so I guess we all know who to point and laugh at when the zombies arrive. When it comes to supplies and bunkers, only 12.5\% had any. However, 68.8\% said that they wished they did. 

Obviously survival is one of the least important aspects of any potential apocalypse, so we started asking the crucial questions, starting with “will you be changing your fashion for the apocalypse / post-apocalypse?”. 31.3\% thought not, but the rest mostly alternated between practical (wearing more layers or camouflage) and punk. Then there’s the one person who boldly stated “Parachute pants, for the inevitable post apocalypse 90s revival”. Sounds fun!

Fashion aside, terminology is also key. General consensus (31.3\%) was that the plural of apocalypse was ‘apocalypses’, but the runner up answer ‘apocalypsicles’ (18.8\%) is much more exciting!! Why not send a letter to the Oxford English Dictionary and see if they can do anything about that…food for thought.

To establish a narrative, we asked responders “Someone on the news just said the apocalypse is happening, and now the world is in chaos. What's the first thing you do?”
The results seemed broadly split between 4 trends:
-	Contact loved ones
-	Find a weapon!
-	Relax or go to sleep 
-	Check secondary sources of information
With the exception of the person who answered “perish” of course. From this we can unreasonably extrapolate that all members of ICSF can be sorted, potentially via hat, into 5 broad categories: caring, violent, chill, pragmatic and dead.

We then went further into detail, asking what weapons our members would make use of in the apocalypse. Some responders opted for conventional apocalypse weapons, such as brass knuckles, baseball bats, machetes, frying pans, Molotov cocktails and hockey sticks, whilst there was also a leaning towards ranged weapons including rifles and bows. A couple of responders seem to view the apocalypse as an opportunity to play out their fantasy genre adventures, as we also had answers of a warhammer and a scythe. Probably the most comedic pair of answers, which when read together make a suspicious amount of sense were:
-	“I would use a comically large spoon. However, after I die, someone would use my corpse as a two-handed weapon”
-	“The chair Phil’s petrified corpse”

Thank you for your sacrifice, Phil…

But what about our beloved Library? What would become of it during the Apocalypse? Well, we asked our responders, and 37.5\% of them thought it would be turned into some kind of bunker or safe haven. However, one person had a less cheerful (though possibly more realistic) view on this outcome:
“A couple of students will shut themselves there and die because they cannot get out. A thousand years later the new people will discover the site and think it was some place of knowledge.”

Another responder theorised, “I feel like it'd be the source, i.e Cthulhu's lair, ground zero for zombies, nuclear strike site priority one”.
The ICSF committee would like to remind everyone, and in particular the person who stated that the Library would be “raided for snacks”, that food is not left in the Library overnight. On an unrelated note, we would like to also ask members not to eat the books or DVDs. They are not silver cookies, and hold no nutritional value (that we are aware of).

In conclusion, it can be assumed that our society would continue in some regard, even if only in the hearts of the surviving third of our members. On a more sincere note, it was really exciting and interesting to see the massive range of answers, from the practical to the bizarre, so thank you to everyone who answered the survey! 
