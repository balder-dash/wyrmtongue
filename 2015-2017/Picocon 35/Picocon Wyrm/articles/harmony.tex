Harmony is an interesting concept, perhaps best expressed as a unity of purpose between well differentiated actors. Whether it's musical notes combining into a chord, or different creative media reinforcing one another, harmony is both about togetherness and celebrating difference. In the natural world, harmony is often a product of chaos rather than order – a balance between elements, as illustrated in Diba's fabulous artwork for this year's convention.

In fiction, often an avenue to explore what we dare not in life, differences between groups are frequently exaggerated, with stubborn dwarves and warlike orcs, or ridge-headed space warriors and large-lobed free-marketeers. Whilst plenty of stories revolve around the conflicting natures of these groups, often influenced by the outlook of those who craft them, there are also stories with a brighter outlook. Time and time again, alliances crop up of humans and pointy-eared creatures, whether they be elves or Vulcans, in which the weaknesses of one are covered by the strengths of the other.

And stepping back from the works themselves, the act of creating rarely occurs in a vacuum. Whether long-standing relationships between illustrations and text, the spoken and the written word, or simply different works inhabiting the same shared universe, the combination of different aspects of creativity come together in broadly harmonious ways. To mix one's proverbs: When standing on the shoulders of giants, no man is an island.