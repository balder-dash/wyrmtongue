A spectre is haunting the corridors of Imperial – the spectre of Picocon.  However, like the vast majority of supernatural occurrences, there is a rational explanation and a large number of people scurrying about behind the scenes aiming to maintain the illusion. All that remains are some plucky guests and a talking dog – though the last bit is not strictly necessary.

This year’s theme is Futurism – a topic rooted in youth, technology and pursuing liberation from the weighty institutions of the past\footnotemark[1]. Our authors have a wide range of written work spanning alternative pasts and future histories, Omega Points, musically inspired horror, and independence-seeking human-machine hybrids. I hope that after listening to their talks throughout the day you will be enthused to borrow or purchase some of their books – our library has something from each of our authors.

The Picocon subcommittee joins me in thanks for all the volunteers who give up their time to ensure that this event continues to be the success that it is – without your efforts, we simply could not support the number of guests and occurrences that make Picocon the high-point of our year. Whilst as a helpful Beanbag I can pick up slack for the glorious Sofa when it comes to internal matters, an army of hands (and accompanying brains) make everything a tad less stressful.

Since I’m sure that this year will continue the trend of seemingly eternal traditions springing precedent-less yet fully-formed from the aether, I would like to end this rambling introduction with a few words of great import in the hope that they will continue to be passed down through the ages: nitwit, blubber, oddment and tweak.

So say we all,

\footnotetext[1]{This liberation excludes Picocon, of course, which remains eternally youthful through regular ritual sacrifice involving paper, pen, and the energy of committee and volunteers in a never-ending quest to attract visitors.}