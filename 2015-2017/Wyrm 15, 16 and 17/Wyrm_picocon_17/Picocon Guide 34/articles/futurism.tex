 Futurism has been defined in many ways – for a while it was an artistic movement aiming to liberate Italy from the mistakes of the past. For a while, it was a relentlessly optimistic dream about a society free from resource constraints, becoming in some places a more cynical view based on extrapolating current societal trends to their presumed conclusion. At heart however, it has always been about us, and about what these trends mean for humanity.

In 2010, the price of the textbook `The making of a fly’ was \$70. For a brief time in 2011, it was \$2.4 million. Rather than this being an art piece with the intent of making some scathing indictment of overpriced textbooks, it was entirely the result of a rather stupid pricing system. Two antagonistic market algorithms escalated and escalated until a stable position was reached – or rather, until the owners noticed the absurdity and shut it down. This is an absurd example, but it highlights the problems with poorly specifying the implementation of an algorithm – a harmless situation in this toy case, but one which has far reaching implications.

The day has come where the existence of machine learning \& artificial intelligence affects more than a single edge case, and instead affects large parts of society. It already affects us.  It affects the institutions and communities we live within, and with advances in medicine, it may in future affect the fundamental qualities of our bodies. If we as a society survive to that point, then it will be rather interesting to see what happens.

A still more glorious dawn awaits?