Dearest attendees, I am your sofa for today.
A sofa, whilst more comfortable than a chair, is not as applicable to every situation. While the chair handles the everyday business, the sofa is brought out to receive honoured visitors. These would be you, dearest attendees.
Welcome to the floor show.

Today has been a work in progress for slightly more than three hundred and sixty five days, with the first strands being drawn together at Picocon 33 a year ago. Whilst last year, we looked behind towards our origins, this year we look towards our future – or the future of those that may come after.  This is both narratively attractive, and relevant. Private space exploration is a very real possibility in the next few years,  with drone landings governed by a CEO that acknowledges the necessity for intrasolar colonisation in the short term. Computers seem to be better at moving small monochrome rocks in specific patterns than people are.  Numbers as large as 21 can now be factorised in polynomial time. The layperson may wonder, what next? Where will we go from here? We may not know yet, but it’s going to be interesting finding out.

We have a plethora of writers for your enjoyment whose works have touched upon topics from unconstrained artificial intelligence to environmentally friendly space colonisation.

Our first speaker, Jaine Fenn, has an extensive history both with science fiction and technology. Her talk will cover the future of science fiction as a medium – and explain ‘Why Sci-Fi Rules the Page’.  Al Robertson’s subsequent talk, `Into heaven, out of hell’, will cover his two books, discussing the inspirations and observations made within them and seeing how the predictions made seem to be turning out given the events between now and the time of writing.  After this, Paul McAuley will discuss his past and current work, and offer some insight into how the saucer books of the 1960s shaped his view on extraterrestrial entities in his talk `Aliens: a short personal history’.  Finally, we have Justina Robson. Her talk, ``Artificial Intelligence: if you want to scare yourself silly, get a future'' is about her journey thinking through both AI and philosophical ramifications - and how that feeds into her books.

Organising this event has been Fun, a word that here means `chaotic and all-consuming’. However, it would not have been possible without my right and left hands, the beanbag and the chair of vice. They have been instrumental in assembling the day that is ahead. I also wish to thank the many volunteers, committee members, and administrative whizzes who have put their time and sanity into this event. Every person in this list has been invaluable, and I am immensely grateful to all of them.

A warm welcome to you all. Live long and prosper.
