
Hi! My name is Noor, this year's Picocon Sofa (more comfortable than a Chair). Having organised events for the society last year as Chair of Vice, I counted myself just about inexperienced enough to attempt a Picocon, and people seemed to think that trusting me with it would be a good plan! I sincerely hope that over the course of the day this decision proves to be at least not totally ridiculous.

When considering the possibe themes of this year's Picocon, I found myself also thinking of the mythos surrounding how the event began. And so this brought me on to the theme of Origins, where I made a pact with myself to attempt to run Picocon 33 while tracing the beginnings; Picocon 1. My investigations are detailed later in Wyrmtongue!

Of course, I also felt that Picocon 33 would not be complete without a pretentious Latin tagline - and so I tried to combine a comon Latin phrase and a pun of sorts to encompass the second aim I had for Picocon 33. Where my first aim was to trace the origins of Picocon, my second aim was to investigate and discuss how storytelling differs across various forms of media. To combine these ideas, `in medias res', meaning in the middle of things, represents how we are currently in Picocon present, have investigated Picocon past, and Picocon future is optimistic (I hope!). Just as the idea of the past, present and future interlink through the annual event of Picocon, in a similar way, various forms of media interlink by their fundamental aim; to tell a story.

With the theme and tagline neatly decided, I moved on to securing Guests of Honour (GoH's) and the myriad of other tasks assigned to my post. With Paul Cornell's experience in different forms of media, Mihelle Paver's Chronicles of Ancient Darkness series set in the Stone Age origins of humanity, and Carrie Hope Fletcher hailing from the West End, War of the Worlds Musical and a soon-to-be fantasy author, I was lucky enough to secure GoH's who allow for a perfect combination required to explore this year's theme and tagline. 

As is traditional to my role, Picocon 33 has gradually grown and eclipsed most of the things in my life, save perhaps the feeling of excitement and mild terror as the day (today!) approaches. However, despite the obvious additional workload, I have thoroughly enjoyed the process of organising this event - from the beginning to the end.

I hope you all have a wonderful time, and I would like to thank everyone for giving me the chance to run Picocon 33.