Once upon a time there lived a young girl in a tent under a cathedral that was slowly falling apart. The tent was one among many others like it. One day, the girl marched up to the front of the church hall, carrying a book bound in tattered leather.

She was around eleven years old, maybe twelve; standing on her tiptoes, she was just about tall enough to meet the eyes of the statue perched upon one of the great pillars that held up the tower. The statue was made of beautiful
white stone: the image of a man draped in robes, kneeling with its head lowered, a pair of wings folded on its back. Its hands were once clasped in front of its face, maybe in prayer; one of the arms had fallen off some time ago.

``I challenge you to a duel,'' said the girl.

The statue opened its eyes a little wider, though the rest of it remained unmoving.

``HM?'' it replied.

``I challenge you to a duel,'' said the girl again.

The statue lifted its head. ``WHY WOULD YOU WANT TO DO THAT?'' It asked.

``Uh,'' the girl began to recount:

``You were mean to Cecilia,'' she said, ``She's my friend. She was looking at you, and you made the glowy circle on your head and now she's blind.''

``You stole from Jacob's dad. You make-pretend that you're angels and you took money from him, and now they don't have enough for food.''

``And you hurt my mum.''

It was easy to tell that she tried to keep her tone level throughout, but her voice so slightly wavered with the last sentence.

``I DO NOT DENY I DID ANY OF THOSE THINGS, EXCEPT ONE,'' said the statue, ``I DID NOT HARM YOUR MOTHER.''

``She was sick,'' said the girl, ``and you did nothing. I know you can heal people, but you didn't help her.''

``I DID NOT HARM YOUR MOTHER,'' the statue repeated, ``I, THROUGH INACTION, ALLOWED HER TO COME TO HARM.''

``Whatever,'' the girl said, unconcerned by the difference.

She bounced a little on her feet, maybe tired from standing on her toes for so long. She sounded impatient. ``Will you fight me or not?''

``HM,'' the statue considered.

``I AM UNARMED,'' it then said. Very slowly, it spread its wings, stone feathers unfurling as if intending to draw attention to the fact that none of those feathers are secretly knives. ``WITH WHAT SHOULD I FIGHT YOU?'' it asked.

``You know \emph{magic,''} the girl pointed out. ``It' s how you heal men who
are sick. It's how you conjure your halo, and other things you do to trick people.''

\emph{Conjure. Halo.} She knew she knew the actual words for make the \emph{glowy circle.} She knew many words, she just needed to work on finding the right ones when she needed them. It's how spells worked---finding the right words and putting them together to make magic. \emph{Words are the source from which all existence springs forth,} her father said often, \emph{to read is to have your eyes unclouded. To write is to have your hands untied.}

\emph{``YOU} ARE UNARMED,'' the statue said next, ``WITH WHAT SHOULD YOU FIGHT ME?''

``I know magic, too,'' said the girl, proud, raising her chin. ``My dad taught me. He's a wizard. He knows things.''

She glanced at the floor. ``Well, not very well,'' she said, ``not anymore. But he wrote it down when he did, and he reads it to me. I figured most of it out.''

She clutched the book a little closer to her chest, and some confidence returned to her eyes.

``I know about the Ones Who Crawl, and about the daemons who live in threads. I know
about the fairies, who deceive,'' she said. ``I know you aren't \emph{really} an angel.''
There was a hint of a taunt in that sentence.

The statue listened. Then, after a moment's contemplation, it rose to its feet.

``FAIR,'' it declared, ``I ACCEPT YOUR CHALLENGE. WHAT IS YOUR NAME, LITTLE ONE?''

``Helen,'' said the girl, ``Helen Lamb.''

And then the girl disintegrated into a puff of haecceity, because she had forgotten the first rule of getting into magic fights: that
you should never give your True Name to a fae creature.
