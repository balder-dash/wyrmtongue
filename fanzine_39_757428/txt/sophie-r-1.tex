There was once a hermit who lived on the light-spire. The name “light-spire” came from a local legend of long-ago lightning striking the earth of a high moor and turning instantly into a twisted, jutting finger of rock. It was on the very tip of the lightning-spire’s nail that the hermit’s hut perched, the place he had his home for most of his life, precariously balanced above a blanket of heather and bracken. 

The hermit purported that his reason for coming to this obscure moor was to become closer to God. But the true reason, the real reason, was that the hermit despised all human contact; totally and completely. His chosen existence was a hard, bleak one, measured in storms, psalms and sheep, but he adapted with the calm self-belief that comes with conviction. Any doubts or pangs of loneliness were silenced ruthlessly and thoroughly by the hermit, as though they were a similar problem to finding clean water or a lost ewe: there was no room for such weaknesses, the hermit believed, not in the vast quietness of the moor and its light-spire. 

From where kith and kin could have surrounded him, the yellow slotted eyes of the sheep regarded him mutely. In the place of warm laughter at the end of an evening, an ever-biting wind moaned as it sharpened its teeth on the edges of the light-spire. 

There was a free-standing world at the tip of the light-spire, defined by many things: the touch of Holy Book’s soft, velvet cover. The smell of fresh cloudberry. The crackling of a winter fire. The curlew’s call. The scratch of a quill. This was the hermit’s life, for decades upon decades. 

Until that night… 

The hermit woke that night with the knowledge something was different. Pulling a sheepskin around his shoulders, he opened the door of the hut, hands stiff with cold. He shook off the mantle of sleep and looked towards the moor. And his breath caught in his throat in a way it hadn’t done for a long, long time: because a sight like no other awaited him. 

Mist. A sea of it, blanketing the moor into invisibility. His hut, a lonely island, stood just shy of the gently roiling surface. 

But the most astounding thing, the thing that had conjured amazement in a soul saturated in solitude, were the angels in the mist. Angels. They had to be; the hermit had no other word for them. Pale purples, greens, blues were diffusing their way through the mist, just like wings slowly beating. It was undoubtably the most beautiful, the most perfect thing the hermit had ever seen. For the time that he spent watching the languid, peaceful motions of the frost-smoke, the hermit forgot why he had decided to come to the moor. Enraptured beyond what he thought could be possible, the hermit watched until the opalescence gracefully faded away into nothingness. 

Overwhelmed, grateful, reverent, the hermit looked up towards heaven to pray. He prayed with his eyes open, looking into the cold glare of the stars, giving thanks with tears in his eyes. But the night had not yet ended. 

Without any warning whatsoever, the \emph{aurora borealis} burst forth once again. 

The angels were no pale reflections in the mist this time. Bright and emerald and violet they shone; a dance of forces marvellous. The hermit felt, right then and there, the truth of his mortality crash down on him, powerful and terrible. He had to clutch his shepherd’s crook tightly to keep from falling over as a heady mix of terror and wonder swirled around inside of him. 

If he had been asked why he suddenly felt that way, the hermit would have replied: “This does not belong to any mortal.” But there was no one to ask him. No one at all, not for leagues and leagues… 

There was no question of the hermit going back to sleep. He stood and waited, washed in the dreadful power of the Northern Lights, awestruck to the very depths of his soul. 

Daybreak, seemingly months later. The clear glow of dawn found the hut on the tip of the light-spire empty and the sheep wandering masterless. Under the gaze of the \emph{aurora}, the hermit had decided to leave behind his life of exile and isolation and start over again in the land of the living. 

For they had found that they did not belong in this place: the light-spire belonged to the angels alone. 
