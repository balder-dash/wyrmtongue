She walked in the darkness through the night, her pace slow as she
journeyed through the land. She slinked through weeds, grass, and
stones, stumbling in the lightless places beneath the canopy. Her
footsteps did not echo, muffled by the softness of the earth beneath
the soles of her sandals. But even without her meagre contribution the
night was far from silent. Eyes were everywhere. The chirps of
crickets synchronised, harsh and cold, a heady beat to their
chanting. A warning, a promise, a threat that sped her along. She knew
it was not only crickets who watched keenly at night, as the ominous
calls of birds hunting for small prey reverberated through the soil
itself. Worse things were out there. But she had to know. She had to
see it with her own eyes.

It was not wise to be out alone at night, especially being a woman
alone in the dark womb of nature. Without a means of
protection. Without aid. But for once the rules of society had to fall
away. Kalpana needed to know. She needed to grieve. And she needed to
do it alone.  She clambered through the thicket, as the trees grew
dense, rough hands slipping over rougher bark. She was getting close
now. The smells were changing from the floral and ripe scents of fruit
to the creamy, subtle touch of old wood. She slipped around two trees
that had grown together into a tight embrace and stumbled into the
meadow.

It had not always been a meadow, and in fact had only become so today
after the logging. The logging of the grand teak tree that had stood
there for hundreds of years. It had stood long before her eldest
sister married. Long before the lives of her parents unfolded. Long
before the land was even her family's.

It was unnaturally wide for a teak tree. It was so wide that when she
and her three sisters had come down to play in their youth, under the
watchful eye of their Acha\footnote{Father.}, they could not span the
girth of the trunk with their eight arms outstretched. They would run
around it, four young girls in colourful cotton dresses, and make up
stories about its past filled with magic. The eldest, Kajal, with a
mole the colour of kohl above her eye, insisted that all teak trees
were imbued with immortality, as the wooden beams of their
grandfather's childhood home were still strong and sturdy. The second
sister, Kamini, the most affectionate among them all, would press her
ear against the tree and declare that they were wrong. No, the tree
was a woman cursed by a jealous demon, but year after year she still
had her beauty as flowers crowned her verdant hair. The youngest
sister, Kajri, whose mind was always in the skies, fancied that the
old teak tree housed the spirits of past clouds, each of which gave up
their lives so the tree could flourish.

The tales made them giggle and laugh, and as the days and years
passed, they spun more elaborate stories of immortals in love, the
endurance of beauty even through the most terrifying encounters, and
the sacrifices made for honour. But no tale caught their imaginations
quite as much as Kalpana's. Kalpana, the third daughter, did not share
the convictions of her sisters. She was not certain that her tale
might touch upon truth, but closing her eyes and clutching the bark, a
gentle hum would tickle her fingertips. Kalpana's story, Kalpana's
\emph{feeling,} was that the tree was not a teak tree at all, despite its
deceptive furry leaves and simple trunk. No, how could it be? It
protected and sheltered them and all the animals that sought peace. It
grew large and strong, casting shade for those in need and
carvingspace for life to bloom. It was splendid all year long, even
when it shed its leaves during the dry season, biding its time to
recover when the monsoons struck. Certainly, that meant it had to be
of sacred origin. Surely, it had to be as sacred as a banyan tree?

Kalpana stopped at the edge of the meadow, her legs rooted to the
ground.

Gone. It was gone.

The absence was palpable, a vast emptiness exuding from the space. The
absence of the wide, towering presence left the flowers and grass
exposed to the elements. The land was freshly worn down. An uneven,
red-clay path lay exposed where the tree had been dragged away, its
heavy corpse clawing at the earth in a last effort to remain where it
belonged. Small branches and twigs lay scattered on either side,
having lost their battles. They were left to die, their future certain
only in that they would become one with the earth again. They would
never soar high above in the breeze or hold firm under the weight of
crows and squirrels again, not unless they were plucked away for a
nest in some faraway place. No, the only thing that would remain where
it had always been was the stump. All else was lost.

Her shoulders heaved against her will, as air pushed and pulled out of
her lungs, straining against her, breaking any sense of control she
had managed to hold together this far. Before it had only been a
terrible piece of news, words as light and empty as the air that spoke
them. It had not been real. Or she had hoped it would not truly be
real. But \textellipsis It was gone. It was really gone.

She stumbled forward with steps uneven as the ground seemed to move
beneath her, no longer as firm and sturdy as she'd always thought. She
was sinking into rising despair, each movement laboured and lost,
until she fell to her knees, grasping the edge of the stump.

It was gone. It was gone. It was gone.

The thought pounded through her skull over and over again, drowning
out any reason, any justification, any voices that might soothe the
ache that overcame her. Her nose tingled in familiar anticipation for
the tears which splashed onto her cheeks. They ran away leaving hot
trails as harsh and empty as the clay path. A wail reverberated
through her chest, lost and confined, just like her. She pressed her
palms over her eyes, willing them to stop, but they would not
obey. They would not listen.

Unable to contain the grief that flooded her system, she whimpered and
clambered atop the stump.

The last shred of a familiar life was gone. The last shred of memories
stolen. The last piece of continuity she had clung onto dying and
drying in the humid heat. All because of a series of bad harvests and
the unjust need for dowries. All these years and memories lost because
it was the only thing they owned of monetary value as the earth and
society abandoned them to their own devices

At first the earth had defied them only occasionally. Sometimes the
harvest was not as good as it should have been, as rains were sparse
and the heat great. The weaker plants wilted under the wilful glare of
the sun and found no solace in trying to bear fruit. Instead of
plenty, whole lengths of crops failed. There had been enough to feed
the family, though Acha and Amma\footnote{Mother.} would often eat
little or nothing at times, but there was barely anything left to be
sold. Close to nothing left to afford their other needs. Eventually,
they sold more and more strips of land, and their cows, and
goats. Amma sought out work as a day-labourer in kitchens, and Acha
worked for others. At least that way there would always be wages,
though not as much as could have been won from a good harvest of their
own.

Kalpana buried her face in her hair, her braid having unravelled. She
cried for what had been, of what was, and what was still to come. She
cried for loss, she cried for fear, she cried for herself. Her tears
would not save the tree now. Her tears would not change the visceral
loss. Her tears would achieve nothing. And yet, all she could do to
stop her very being from breaking into shards of clay was cry.

The call of the crickets disappeared into nothingness, and the yells
of all the hunters of the night faded into the recesses of her
mind. All that remained was the familiar embrace of the leather scent
of the tree as she curled up on the stump.

`Kalpana,' her father's familiar voice exclaimed.

Her breath hitched, breaking the monotony of her quiet sobs. She'd
been found out. Maybe Acha hadn't seen her yet\textemdash maybe she
could hurry home\textemdash

She pushed herself up, mentally preparing to swing her legs over the
edge of the stump when she opened her eyes.

Her charcoal eyes grew wide.

The moon had crept out from behind the heavy clouds that kept it
company and brought with it a pillar of life. A life which took on the
shape of the tree's bark, surrounding her from every angle in pale
milky light. It grew from the skies, the branches wide and bare,
fusing down into the shell of the trunk, until it became one with the
stump below. She was fully encased in the softly glowing,
near-transparent bark. The detailed scratching of the bark gave the
outer world a near ethereal sheen, bending and distorting the world
ever so slightly. It was a world both familiar and foreign at
once. She reached forward, drawn to touch the strange material-

`Kalpana!' Acha exclaimed from behind, and she spun around.

Acha stood at the far end of the clearing, where the trail of the tree
should have lain, but the red groove was nowhere to be seen. She stood
still, mouth frozen mid-breath. He rushed in her direction, striding
towards her through the thicket to the best of his ability. Only as he
drew closer did she notice that his hair was not the salt and pepper
that she had grown accustomed to, but a deep black. His face looked
different, skin younger and fresh. The lines of worry that had etched
into his face were different. Not as deep, though they were still
there.

`I told you to stay by my side,' he exhaled in relief, falling to his
knees before the tree, staining his white mundu\footnote{A garment
  worn around the waist by men in parts of India.} with red soil as he
held his arms wide open. But his eyes weren't focused on her. She
looked down, and saw by the edge of the tree, a little girl, barely
three years old, sucking her thumb. The girl looked at Acha, her young
mind not understanding the strange reaction of her father, and ran
over to him, hugging him tight.

`But tree so big,' she exclaimed, her wild curls bouncing around her
oval face as she giggled.

`I know it's big,' he sighed softly, clutching her against his
chest. `But I need you to stay with me, little one. What would I do if
I lost you? What would I do if I lost any of you girls?'

The little girl burst into peals of laughter and snuggled up to her
father.

`Never lost, Acha. Always here.'

Kalpana's breath hitched. She'd never been able to recall what had
happened that day, but knew the story so well. She had run away while
Acha was taking care of their cow, and he had searched for her
everywhere, terrified that a wild dog might have found her. But she
had been fine. Perfectly unharmed child simply toddling along.

Instinctively, Kalpana leant forward, reaching towards them. When her
fingertips touched the milky surface, the tree rippled in its
wake. Acha and the little girl disappeared, fading away as quickly as
they'd arrived.

`You can't be serious,' Amma huffed, leaning against another side of
the trunk, clutching against the wood, her fingers bent with the
strain of it.

`There is no other way,' Kajal said, running a hand through her hair
as she rested an empty basket against her hip. `You and Acha are not
getting any younger, and this way you won't have to deal with the
burden of feeding four of us.'

`But you're still so young!' Amma protested, squeezing her eyes shut
for a moment, a heaviness overcoming her mind. `Too young to know your
mind. Too young to understand his mind\textemdash You barely know
this boy\textemdash'

Kajal laughed, her eyes twinkling in the spectre of sunlight, simply
shaking her head.  `I am 24, Amma. Three years older than you were
when you married Acha.'

`It's not the same,' Amma protested, an edge of despair in her
voice. Deep wrinkles carved around her mouth, as her lips set in a
deep frown.

`Why is this any different?'

Amma reached forward instinctively, but her hand froze mid-motion,
curling into a small fist instead. She placed it back on the tree and
looked away.

`That was myself,' she whispered, hands curling, trying to find
purchase in the tree. Trying to find anything to hold onto.

`So?' Kajal asked softly.

Amma swallowed, and squeezed her eyes shut for a long moment, every
fibre of her shaking.

`I didn't lose anything then,' she said, her voice breaking.

`Oh, Amma\textellipsis' Kajal threw the basket aside, and rushed to her mother's
side, pulling her into a tight embrace. She buried her face against
her mother's shoulder and shook her head. `You will never lose me.'

`That's not true,' Amma whispered, clutching her Kajal's
churidar\footnote{Common dresses worn by women of the Indian
  subcontinent.}, and bunching it in her hands.

`I am always going to be there for you, and Acha,' Kajal
promised. `Same for Kamini, Kalpana and, even Kajiri, when she's not
being annoying,' she chuckled, eliciting a small laugh from Amma. She
smiled warmly, pressing a kiss to her mother's cheek. `We never truly
lose the people we love.'

\emph{That's not true,} Kalpana thought, a bitter taste in her mouth.

The vision rippled again, making Amma and Kajal disappear once more.

`Wait!' Kamini called from across the meadow, her waist-length hair
billowing behind her as she ran, waving her arm above her head. She
waved at a group of men who stood to the side of the tree, axes and
tools in hand, waiting to be raised. Behind them stood Acha and the
contractor who had visited their tin-roofed home a couple times over
the last weeks. `Please, Acha, tell them to wait!'

He spoke quietly to the contractor, who raised his hand in a sign of
pause. The workers shrugged and lowered their axes, beginning to talk
amongst each other again, though their eyes were trained on the
flushed young woman.

Acha walked towards the tree, away from the group, and waited for
Kamini to arrive.

`Why did you come?' He whispered calmly, though his eyes gave away an
urgency that he didn't want the workers to see. There was a sternness
in his countenance that few would recognise. `You know there is no
other way forward. We discussed this already.'

`I know,' she panted, gulping for air. She placed her hand on the
tree, and the spot where she touched it glowed, `but I needed
to\textemdash'

She squeezed her eyes shut and took a deep breath, trying to steady herself.

`Needed to what?' Acha prompted.

She exhaled in a short huff, her cheeks red with exhaustion. The run
had not been trivial, though she knew the land as well as she knew her
own hand. Tears stung her eyes as she looked up at Acha.

`I need to say goodbye,' she whispered, her voice breaking.

The sternness melted away, and Acha's moustache twitched. His eyes
softened and he nodded softly. She did not need to say more. He led
her further around the tree to a spot diametrically opposed to the
workers and squeezed her shoulder.

`Say what you need to,' he whispered. `We will wait.'

Her lips trembled, as she clutched his hand, and nodded, watching as
he walked back towards the workers. He spoke to them, but the sound
was muffled as if travelling through honey. The world blurred until
only Kamini remained in focus.

She lay both her hands on the bark, pinpricks of light glowing from
the contact. She leant close, looking directly at the tree.

`I'm so sorry we're doing this to you,' she whispered, eyes brimming
with unspent tears. `This is all our fault. We have no right to demand
this sacrifice from you. To take you from your home. From where you
have always belonged\textellipsis' She looked down, teardrops spilling onto its
roots. Each stain glowed intense white. `I'm sorry,' she laughed, a
hopeless edge to her smile. `I keep watering you with my silliness… I
just…' Her face crumbled, and she pressed her cheek against the tree
and hugged it. `I just hope you know we love you and thank you for
saving us.' She stroked the bark slowly, like someone might stroke the
hand of a dying loved one. `We never forget the ones we love.'

Kalpana gasped as the tree started pulsing. The light spread from
where Kamini had touched it, and moved outwards, fanning down into its
roots and up into its branches, over and over. The empty ghostly
branches grew outwards, leaves and flowers budding and unfurling,
releasing a sweet fragrance that wound itself around Kalpana's entire
being.

It was an expression of love. A feeling shared and treasured even
after its physical body had been torn in two. A promise that it too
would not forget.

The branches shook slowly, swaying gently like there was a
breeze. White flowers cascaded, blessing the earth like warm rains and
sunshine. She watched it all with wonder, gasping at the vision so
unlike anything she had ever witnessed.

It was acceptance, it was love, and it was goodbye.

Eventually the branches stopped and the glow began to dim. It was slow
and gradual, but the light bled away until all that was left was
Kalpana, standing on a stump, bathed in moonlight.

She slowly stepped down, finding a firm footing at last. She walked
towards the edge of the meadow, and turned one last time to whisper
goodbye, knowing with certainty that it heard her and understood.
