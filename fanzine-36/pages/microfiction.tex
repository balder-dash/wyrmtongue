
\newcommand{\microfiction}[3]{
  \vspace{\baselineskip}
  {\Large #2} \\[-0.6em] \begin{flushright}
    \textemdash{} \textit{#1, #3 characters}
  \end{flushright}
  \vspace{\baselineskip}
}
%% \newcommand{\entrysep}{
%%   \vspace{1.0\baselineskip}
%%   \begin{center}\rule{0.1\textwidth}{0.4pt}\end{center}
%%   \vspace{1.5\baselineskip}
%%}

\begin{center}
  \ctth{Microfiction}

  \begin{minipage}{0.72\textwidth}

    \vspace{1\baselineskip}

    This year, we asked our members to try and tell the best story
    they could within a 140-character limit. Here's a selection from
    what they came up with:

    \shorthline
    \microfiction{Kai Lawrence}{ The hero always gets the credit, never
      me. It's always Thor's Mjölnir or Arthur's Excalibur. Maybe I should
      quit mage-forging?}{126}

    \shorthline
    \microfiction{Smitha Maretvadakethope}{At first it was one
      colony. Wiped from existence by a flood which burst all seams. But
      the creator continued raising and razing her \textit{E coli.}}{138}

    \shorthline
    \microfiction{Erin Lovell}{Keep jewellery away from pet lizards: they
      grow wings, learn to breathe fire, once they amass a hoard worth
      protecting as a dragon.}{131}

    \shorthline
    \microfiction{Yeety McYeetface\footnotemark}{With the wide acceptance of
      superpowers, the College Union starts creating new health
      and safety rules. No. 1: No being on fire in the libr}{139}

  \end{minipage}
\end{center}

\footnotetext{The submissions were received through digital forms, and
  the author credit on this entry is evidence that we have learnt
  nothing about the internet.}
